\section{Random Assignment}
\subsection{Treatment Randomly Assigned to Students and Teachers?}

Balance holding across grades should imply randomness; if we fail to reject the null, there is no statistical difference between the beta coefficients. This suggests that the differences in outcome variables (e.g., school lunch, race, and age) are random—meaning they are independent of class-size assignment. 

However, this statistical insignificance does not prove randomness; it only indicates that we cannot reject the possibility of it. True randomness is derived from the experimental design itself, not merely from the statistical results. While these results provide increased confidence in the study's integrity, they come with caveats: based on the F-test results, we cannot reject the null, meaning randomness is possible, but this conclusion remains a statistical inference rather than an absolute certainty.
