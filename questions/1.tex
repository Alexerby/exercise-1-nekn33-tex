\section{Descriptive Statistics}

\subsection{Generating Variables}
See attached STATA code.

\subsection{Descriptive Statistics of New Variables}



Table \ref{tab:combined_star} presents the means of student characteristics across the three treatment arms: small classes, regular classes, and regular classes with a teacher's aide. 
The final column reports the $P$-value from a joint $F$-test for the equality of means across all three categories.


\subsection{Testing for Randomization (Balance Test)}

To see if the randomization actually worked, we need to check if the student groups looked the same at the starting line. We test the null hypothesis ($H_0$) that there are no significant differences in the means of baseline characteristics, specifically Free Lunch, Race, and Age, across the different treatment groups. 

To do this, we run a separate OLS regression for each characteristic. Each row in Table \ref{tab:combined_star} essentially summarizes one of these models. For example, the row for ``Free Lunch'' represents the following regression:
\begin{equation}
    \text{Free Lunch}_i = \underbrace{\beta_0}_{\text{Small Mean}} + \beta_1(\text{Regular}_i) + \beta_2(\text{Regular+Aide}_i) + \epsilon_i
\end{equation}
For each model, we perform an $F$-test for the joint significance of the treatment indicators ($H_0: \beta_1 = \beta_2 = 0$). 
As long as we fail to reject this null (i.e., we get a high $p$-value), we can conclude that the groups are ``balanced''. 
This means no specific class type was accidentally stacked with students of a certain background, ensuring our different class sizes have similar pre-treatment status.


\subsubsection{Interpretation of Results}
The results generally support the claim that the initial assignment in Kindergarten was random. In Panel A, the $P$-values for Free Lunch (0.246), Race (0.238), and Age (0.341) are all well above the standard 5\% significance level. This indicates that we fail to reject the null hypothesis of balanced characteristics, suggesting that students were distributed across class types without systematic bias.

However, as the experiment progressed into later grades (Panels B, C, and D), several $P$-values for Race and Age became statistically significant ($P < 0.05$). While the lack of a clear, consistent pattern across all variables suggests that balance was maintained in many dimensions, these significant $F$-values indicate that the data is not perfectly random in later years. This potential "unbalancing" is likely due to non-random attrition or student mobility between class types after the initial Kindergarten assignment. Despite these exceptions, the overall results for Kindergarten provide strong evidence for the integrity of the initial experimental design.


\begin{table}[htbp]
    \centering
    \caption{Student Characteristics and Joint $P$-values by Grade}
    \label{tab:combined_star}
    \def\sym#1{\ifmmode^{#1}\else\(^{#1}\)\fi}
    \small
    \begin{tabular}{lcccc}
        \toprule
        Variable & Small & Regular & Regular+Aide & Joint $P$-value \\
        \midrule
        \multicolumn{5}{c}{\textbf{Panel A: Kindergarten}} \\
        Free Lunch                       & 0.468 & 0.480 & 0.495 & 0.246 \\
        White/Asian                      & 0.686 & 0.676 & 0.660 & 0.238 \\
        Age in 1985                      & 5.257 & 5.238 & 5.239 & 0.341 \\
        Average Score                    & 465.803 & 458.943 & 459.616 & 0.000 \\
        Class size                       & 15.396 & 22.375 & 23.257 & 0.000 \\
        \addlinespace
        \multicolumn{5}{c}{\textbf{Panel B: Grade 1}} \\
        Free Lunch                       & 0.438 & 0.468 & 0.440 & 0.220 \\
        White/Asian                      & 0.689 & 0.630 & 0.704 & 0.000 \\
        Age in 1985                      & 5.333 & 5.423 & 5.424 & 0.000 \\
        Average Score                    & 534.646 & 519.456 & 525.740 & 0.000 \\
        Class size                       & 15.690 & 22.689 & 23.420 & 0.000 \\
        \addlinespace
        \multicolumn{5}{c}{\textbf{Panel C: Grade 2}} \\
        Free Lunch                       & 0.411 & 0.428 & 0.406 & 0.587 \\
        White/Asian                      & 0.684 & 0.645 & 0.652 & 0.033 \\
        Age in 1985                      & 5.428 & 5.517 & 5.509 & 0.000 \\
        Average Score                    & 588.757 & 578.729 & 581.359 & 0.000 \\
        Class size                       & 15.310 & 23.464 & 23.469 & 0.000 \\
        \addlinespace
        \multicolumn{5}{c}{\textbf{Panel D: Grade 3}} \\
        Free Lunch                       & 0.383 & 0.391 & 0.405 & 0.614 \\
        White/Asian                      & 0.690 & 0.677 & 0.656 & 0.075 \\
        Age in 1985                      & 5.487 & 5.545 & 5.554 & 0.003 \\
        Average Score                    & 621.863 & 614.685 & 614.548 & 0.000 \\
        Class size                       & 15.714 & 23.610 & 24.435 & 0.000 \\
        \bottomrule
    \end{tabular}
    \vspace{0.2cm}
    \begin{flushleft}
        \footnotesize \textit{Note:} Joint $P$-values refer to the $F$-test for equality of means across the three class types. Average score is the mean of reading and math scaled scores.
    \end{flushleft}
\end{table}

Comparing Table \ref{tab:combined_star} in this paper with the original Table 1 from \textcite{krueger1999}, we find that our results are mostly the same but vary slightly due to differences in the dataset.
