\section{External Validity}
\subsection{Hawthorne and John Henry Effects}
\textcite[526-527]{krueger1999} brings up in this paper that it has been suggested by some that the effectiveness of small classes in the STAR experiment is due to Hawthorne (HT) and/or John Henry (JH) effects
\begin{itemize}
    \item \textbf{Hawthorne:} A behaviour where individuals who are observed modify their behaviour, due to the motivation gained by the interest that is shown to them.
    \item \textbf{John Henry:} A behaviour where individuals who are aware that they are in the control group will work harder to outperform the treatment group, skewing results.
\end{itemize}

In the context of the STAR experiment, the concern is that the observed test score gap might not be a real result of class size, but instead a psychological byproduct. 
Teachers in small classes might have worked harder because they felt special for being in a study (Hawthorne), while teachers in regular classes might have pushed themselves more to prove they could still succeed despite having more students (John Henry). 

\subsection{Addressing Hawthorne and John Henry}
To address concerns that the STAR results were driven by the psychological reactions of teachers (Hawthorne or John Henry effects), Krueger conducts a falsification test by examining variation strictly within the control group. The logic is that Hawthorne and John Henry effects are ``reactive'' biases, they only occur when a teacher knows they have been assigned to a special treatment or a disadvantaged control arm.

Krueger regresses student test scores on the actual class size using only the sample of students assigned to regular-size classes. 
Because every teacher in this sample is in the control group, their ``experimental status'' is identical. 
If the test score gap was caused only by psychological reactions to the experiment, class size should have no predictive power within this group. 
However, Krueger finds a statistically significant relationship (t-statistic of -4.3), where even small reductions in class size among regular classes lead to higher scores. 
This suggests that the class-size effect is a real pedagogical phenomenon rather than a byproduct of the experiment's design.

\subsection{Replication Krueger Falsification Test}
When \textcite{krueger1999} performed the test described in the previous subquestion, he found a coefficient of -.55 with a \( t \)-ratio of -4.3.
In Table \ref{tab:hawthorne_check} we replicate the falsification test and find a similar outcome to Kruegers, in our case a coefficient of -.52 with a \( t \)-ratio of -4.5.

The reason ours is not identical is due to some small deviation in the model specification or the dataset itself.


\begin{table}[htbp]
\centering
\caption{Replication of Hawthorne/John Henry Sensitivity Check (Control Group Only)}
\label{tab:hawthorne_check}
\begin{tabular}{lcc}
\hline\hline
\textbf{Dependent Variable: Pooled Test Score} & \textbf{Coefficient} & \textbf{(Std. Error)} \\
\hline
Actual Class Size (\textit{csize}) & -0.515*** & (0.116) \\
\\
\textbf{Student Characteristics} & & \\
White/Asian Dummy & 22.125*** & (1.212) \\
Girl & 6.313*** & (0.621) \\
\\
\textbf{Teacher Characteristics} & & \\
Teacher Experience (Years) & -0.006 & (0.040) \\
Teacher Higher Degree Dummies & Included & -- \\
\\
\textbf{Fixed Effects} & & \\
Grade Level Dummies & Included & -- \\
School Fixed Effects & Included & -- \\
\hline
Observations & 12,540 & \\
R-squared & 0.384 & \\
\textit{t}-statistic on Class Size & -4.45 & \\
\hline\hline
\multicolumn{3}{l}{\small Robust standard errors in parentheses. *** p$<$0.01, ** p$<$0.05, * p$<$0.1} \\
\end{tabular}
\end{table}

Next up, we perform the exact same test but using only the treatment group in contrast to only using the control group. 
The results for this test is compared to the one we just performed in Table \ref{tab:comparison} and shows that the class size is no longer significant (\( t \)-stat -0.34).
The main message here being that class size seems to be less important once the number of students is low enough. 
It does not matter if the class has 13 or 14 students, but going from 23 to 24 students does matter for the test scores.

\begin{table}[htbp]
\centering
\caption{Within-Group Sensitivity Analysis: Control vs. Treatment Arms}
\label{tab:comparison}
\begin{tabular}{lcc}
\hline\hline
 & \textbf{(1) Control Group} & \textbf{(2) Treatment Group} \\
\textbf{Variables} & \textbf{Test Score} & \textbf{Test Score} \\
\hline
Actual Class Size (\textit{csize}) & -0.515*** & -0.110 \\
 & (0.116) & (0.326) \\
\\
Student \& Teacher Controls & Yes & Yes \\
Grade \& School Fixed Effects & Yes & Yes \\
\hline
Observations & 12,540 & [Check your Obs count] \\
R-squared & 0.384 & [Check your R2] \\
\textit{t}-statistic & -4.45 & -0.34 \\
\hline\hline
\multicolumn{3}{l}{\small Robust standard errors in parentheses. *** p$<$0.01, ** p$<$0.05, * p$<$0.1} \\
\end{tabular}
\end{table}
