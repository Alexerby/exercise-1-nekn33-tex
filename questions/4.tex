\section{External Validity}
\subsection{Hawthorne and John Henry Effects}
\textcite[526-527]{krueger1999} brings up in this paper that it has been suggested by some that the effectiveness of small classes in the STAR experiment is due to Hawthorne (HT) and/or John Henry (JH) effects
\begin{itemize}
    \item \textbf{Hawthorne:} A behaviour where individuals who are observed modify their behaviour, due to the motivation gained by the interest that is shown to them.
    \item \textbf{John Henry:} A behaviour where individuals who are aware that they are in the control group will work harder to outperform the treatment group, skewing results.
\end{itemize}

In the context of the STAR experiment, the concern is that the observed test score gap might not be a real result of class size, but instead a psychological byproduct. 
Teachers in small classes might have worked harder because they felt special for being in a study (Hawthorne), while teachers in regular classes might have pushed themselves more to prove they could still succeed despite having more students (John Henry). 

\subsection{Addressing Hawthorne and John Henry}
To address concerns that the STAR results were driven by the psychological reactions of teachers (Hawthorne or John Henry effects), Krueger conducts a falsification test by examining variation strictly within the control group. The logic is that Hawthorne and John Henry effects are ``reactive'' biases, they only occur when a teacher knows they have been assigned to a special treatment or a disadvantaged control arm.

Krueger regresses student test scores on the actual class size using only the sample of students assigned to regular-size classes. 
Because every teacher in this sample is in the control group, their ``experimental status'' is identical. 
If the test score gap was caused only by psychological reactions to the experiment, class size should have no predictive power within this group. 
However, Krueger finds a statistically significant relationship (t-statistic of -4.3), where even small reductions in class size among regular classes lead to higher scores. 
This suggests that the class-size effect is a real pedagogical phenomenon rather than a byproduct of the experiment's design.
